\documentclass[class=book, crop=false]{standalone}
\usepackage[subpreambles=true]{standalone}
\usepackage{import}
\usepackage[utf8]{inputenc}
\usepackage[margin=1.2in]{geometry}
\usepackage[sorting = none,
            doi = true  %lesedato for url-adresse
            ]{biblatex} %none gir bibliografi i sitert rekkefølge
\addbibresource{reference.bib}
\usepackage{csquotes}
\usepackage{pgfplotstable}
\usepackage{booktabs}
\pgfplotsset{compat=1.15}

\begin{document}
\section{Programming software}
Python is the programming language used for this solving this task. Python is an open-source software that has several packages for machine learning and power flow calculations \cite{python_web}. In this thesis, \textit{pandapower} is the python package used for defining an electric power system and preforming power flow calculations\cite{pandapower}. \textit{Keras} is the Python package used for training a reinforcement agent \cite{keras_chollet2015}.

\section{Pandapower elements}
This section will describe the different electrical elements available in pandapower and how they are physically modelled.
\subsection{Lines}
Table \ref{tab:standard_lines_pandapower} shows the available standard line that can be chosen with corresponding parameter values. Custom parameters can also be specified using the method \texttt{pandapower.create\_line\_from\_parameters}
\begin{table}[]
    \centering
    \pgfplotstabletypeset[
col sep = comma,
string type,
every head row/.style={before row=\toprule,after row=\midrule},
every last row/.style={after row=\bottomrule},
display columns/0/.style={string type,column name={}}
]
{data/linetypes.csv}
    \caption{Standard lines available in pandapowers with values for resistance \textit{r}, inductance \textit{x}, capacitance \textit{c}, max current \textit{I} and cross sectional area \textit{Q} and line type. “ol” is overhead line and “cs” is underground cable system}
    \label{tab:standard_lines_pandapower}
\end{table}
The line are modelled using the $\pi$-equivalent model, mentioned in section \ref{section:pi_model}.

\subsection{Generators}
Pandapower has two generator types. The first type is simply called generator and is modelled as a PV-bus. In other words, the active power production $P$ and voltage magnitude $|U|$ is are known when solving the power flow equations. Generators can be created using the method \texttt{pandapower.create\_gen}, where nominal values for apparent power and voltage can be specified. The second type is the static generator, where the active power $P$ and reactive power $Q$ is specified (PQ-bus). Static generators are created using \texttt{pandapower.create\_sgen}. Pandapower models power from the consumer perspective, so negative values for active power $P$ corresponds to generation of power.

\subsection{Loads}
The load in pandapower is modelled as a PQ-bus, where active and reactive power are known. Loads are created using the method \texttt{pandapower.create\_load}. The loads can also be modelled with constant impedance $Z$, current $I$ and $P$. In other words, replacing reactive power with current and impedance. 

\subsection{Transformers}










\end{document}

