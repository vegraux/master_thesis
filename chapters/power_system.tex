\documentclass[class=book, crop=false]{standalone}
\usepackage[subpreambles=true]{standalone}
\usepackage{import}
\usepackage{amsmath}
\usepackage[margin=1.2in]{geometry}
\usepackage[sorting = none,
            doi = true  %lesedato for url-adresse
            ]{biblatex} %none gir bibliografi i sitert rekkefølge
\addbibresource{reference.bib}
\usepackage{csquotes}

%\usepackage{float}

\begin{document}



%\chapter{Electrical power system}
The reinforcement agent is working within an electrical power system and it is therefore necessary to give an introduction to the electrical grid and relevant quantities describing it. 

\section{Electric circuit theory}
\subsection{Voltage, current and power}
An electric circuit is a model that describes how electrical power is transferred from a electrical source unit to a load. An example of a source is a power socket on the wall and a load can be a light bulb or a vacuum cleaner. A model of a simple electric circuit is shown in figure \ref{fig:theory:simple_circuit}, where \textit{U} is placed next to the electrical source and \textit{R} next to the load. 

\begin{figure}[ht!]
    \centering
    \subimport{../}{circuits/simple_circuit.tex}
    \caption{Simple electric circuit with voltage source \textit{U}, current \textit{I} and resistance \textit{R}.}
    \label{fig:theory:simple_circuit}
\end{figure}



\textit{U} is the voltage in the circuit and is a measure of the potential energy between charges at each end of the voltage source. Volt $V$ is the unit for voltage, which is equivalent to joule per coulomb. The current \textit{I} flowing in the wire is a measure for the amount of charges passing through a cross section of the circuit wire per second. The unit for current is ampere \textit{A} or coulomb per second. The resistance \textit{R} is a measure for how much an electric load, such as a light bulb, resists the flow of electric charges. The unit for resistance is ohm denoted by $\Omega$. The magnitudes of the voltage, current and resistance are governed by Ohm's law

\begin{equation}\label{eq:theory:ohm_simple}
    U = RI
\end{equation}

where  \textit{U} is the voltage, \textit{R} is the resistance and \textit{I} is the current flowing. Another version of this is equation is found by introducing the admittance $Y$ which is defined as the inverse of the resistance $R$, i.e  $Y = 1/R$. The admittance $Y$ is a measure for how a load allows the flow charges in a circuit. The unit for admittance is siemens $S$. By using the admittance $Y$, Ohm's law can be expressed as
\begin{equation}\label{eq:theory:ohm_simple_inverse}
    I = UY
\end{equation}

The power $P$ an electric load consumes can be found easily by multiplying the current \textit{I} and voltage drop \textit{U} over the load
\begin{equation}\label{eq:theory:apparent_power}
    P = UI
\end{equation}
The unit for power is watt \textit{W} or joule per second.

\subsection{Kirchoff's laws}
Figure shows a circuit with several branches and loads 

\begin{figure}[ht!]
    \centering
    \subimport{../}{circuits/simple_circuit_branches.tex}
    \caption{Simple electric circuit with several branches and resistors}
    \label{fig:theory:simple_circuit_branches}
\end{figure}

The current \textit{I} will split when it reaches and intersection, in such a way that the total current flowing into the node equal the total total current flowing out from a node. Referring to figure \ref{fig:theory:simple_circuit_branches} we have that $I = I_1 + I_2$. This conservation of current is called Kirchoff's 1st law or simply kirchhoff current law (KCL). With the introduction of branches in a circuit, there will also be closed loops. In figure \ref{fig:theory:simple_circuit_branches} there are two closed loops $C_{1}$ and $C_{2}$. Kirchoff's 2nd law , aslo known as Kirchoffs's voltage law (KVL), states that the voltage \textit{U} over the components in a closed loop $C$ is equal to zero. 

\begin{equation}\label{eq:theory:kirchoffs_2nd_integral}
    \sum_{i}^{N} V_{i} = 0, \;\; V_{i} \in C
\end{equation}
For the two loops $C_1,C_2$ in figure \ref{fig:theory:simple_circuit_branches} we have 


\begin{equation}
   \begin{aligned}\label{eq:theory:krirchoffs_2nd_ex}
    U + I_{1}R_{1} = 0 
    \\
    -I_{1}R_{1} + I_{2}R_{2} = 0
\end{aligned} 
\end{equation}

\section{Reactive components}
There are electrical loads that cause a phase shift between the current and voltage in an electric circuit. For instance, the circuit shown in figure has a capacitor as load. A capacitor is a component that can store an electrical charge \textit{Q} when a direct current (DC) voltage source \textit{U} is applied over it. A capacitor is characterised by how much charge \textit{Q} it can hold for a given DC voltage \textit{U}. This quantity is called the capacitance \textit{C} of the capacitor and is given by 

\begin{equation}
   \begin{aligned}\label{eq:theory:capacitance}
C = \frac{Q}{U}
\end{aligned} 
\end{equation}
where \textit{Q} is the charge stored by the capacitor and \textit{V} is the applied DC voltage. The units for capacitance is farad \textit{F}. Applying Kirchoff's voltage law to this circuit when the source is a sinusoidal signal $Usin(\omega t)$ and using equation \eqref{eq:theory:capacitance} gives



\begin{equation}
   \begin{aligned}\label{eq:theory:capacitance_diffeq}
U\sin{(\omega t)} - \frac{Q(t)}{C} = 0, \;\;\; Q(0) = 0
\end{aligned} 
\end{equation}

Recognising that the current \textit{I} is the derivative of \textit{Q} with respect to time reveals that the the current is given as

\begin{equation}
   \begin{aligned}\label{eq:theory:capacitance_diffeq_solved}
I(t) = U\omega C \sin{(\omega t + \pi/2)}
\end{aligned} 
\end{equation}
The solution shows that the current \textit{I} is phase shifted 90 degrees ahead of the voltage \textit{U}. It is convenient to introduce the capacitive reactance $X_{c}$ as



\begin{equation}
   \begin{aligned}\label{eq:theory:reactance_capacitive}
X_{c} = \frac{1}{\omega C}
\end{aligned} 
\end{equation}
where $\omega$ is the angualar frequency of the signal and \textit{C} is the capacitance of the capacitor. The circuit current can now be expressed on the same form as Ohm's law

\begin{equation}
   \begin{aligned}\label{eq:theory:capacitance_diffeq_solved_ohm}
I(t) = \frac{U}{X_{c}} \cdot sin(\omega t + \pi/2)
\end{aligned} 
\end{equation}

The circuit shown in figure is another example of a reactive component that phase shifts the current. The load is an electromagnetic coil, also called an inductor. The voltage  across an electromagnetic coil is proportional to the derivative of the current flowing through it due to Faraday's law of induction. The proportional constant is called the inductance \textit{L} of the circuit and it is given in henry \textit{H}. Similarly as with the capacitor, Kirchoff's voltage law  gives rise to a differential equation. Let the voltage source be $U\sin{(\omega t)}$

\begin{equation}
   \begin{aligned}\label{eq:theory:inductance_diffeq}
U\sin{(\omega t)} - LI'(t) = 0, \;\;\; I(0) = 0
\end{aligned} 
\end{equation}
The solution of equation \eqref{eq:theory:inductance_diffeq} is

\begin{equation}
   \begin{aligned}\label{eq:theory:inductancee_diffeq_solved}
I(t) = \frac{U}{\omega L} \cdot sin(\omega t - \pi/2)
\end{aligned} 
\end{equation}


The current is shifted 90 degrees behind the voltage in this case. We see that both inductors and capacitor shift the current by 90 degrees, but in different direction. The inductive reactance is defined as

\begin{equation}
   \begin{aligned}\label{eq:theory:reactance_inductive}
X_{c} = \omega L
\end{aligned} 
\end{equation}
where $\omega$ is the angular frequency of the voltage and $L$ is the inductance of the coil. 

\section{Reactive power}
For a purely resistive load with no phase shift, the instantaneous power transferred to the load as given by \eqref{eq:theory:apparent_power} is always positive. This power is called active power, and is measured in watt W. This is not the case for reactive loads. The phase shift between current and voltage results in a pulsating power between the source and the load, as shown in figure \ref{fig:theory:reactive_powers}. The pulsating power resulting from reactive loads is called reactive power \textit{Q} and has unit var, to distinguish it from unidirectional power flow. Figure \ref{fig:theory:reactive_powers} shows that the reactive power resulting from an inductive and capacitive load are opposite of each other. As a consequence, a circuit with equal inductive and capacitive reactance connected in parallel will draw zero power from the source. By convention, a capacitive load is defined to supply reactive power while an inductive load is a reactive power drain. 

\begin{figure}[ht!]
    \center
    \subimport{../}{circuits/plot_powers.tex}
    \caption[size = 9]
    {Power transferred to the load in a circuit with a pure resistive, inductive and capacitive load that are equal in size}\label{fig:theory:reactive_powers}
\end{figure}


\section{Voltage, current and power as complex numbers}
Current, voltage, impedance and and power are all expressed as complex numbers in an AC electric power system. Consider the circuit shown in figure \ref{fig:theory:complex_current_circuit}
\begin{figure}[ht!]
    \center
    \subimport{../}{circuits/complex_currents_circuit.tex}
    \caption[size = 9]
    {Circuit with a resistor and a inductance connected i parallel}\label{fig:theory:complex_current_circuit}
\end{figure}
The resistor will draw a current $I_{p}$ that is in phase with the source voltage. The inductor will draw a current $I_{q}$ that lags the voltage by 90 degrees. The resultant current drawn from the source will therefore be a linear combination of two phase shifted sinusoidal signals. Euler's formula states that a complex number $A$ can be expressed by

\begin{equation}\label{eq:eulers_equation}
    A = Re^{j\omega t} = R\cos{(\omega t)} + jR\sin{(\omega t)}
\end{equation}

where $e$ is the base of the natural logarithm and $j$ is the imaginary unit. The currents $I_{p}$ and $I_{q}$ signal can therefore be expressed as the imaginary part of two complex numbers. Treating the currents as complex numbers makes it easier to sum them because they form a right angled triangle, as shown in figure \ref{fig:theory:phase_current}. One can always convert back to the sinusoidal current by taking the imaginary part of the complex resultant current \textit{I}.

\begin{figure}[ht!]
    \center
    \subimport{../}{circuits/phase_currents.tex}
    \caption[size = 9]
    {Current as complex numbers. $I_{q}$ is the current drawn by an inductor, and lags the voltage by 90 degrees. $I_{p}$ is the current drawn by a resistive load and is in phase with the voltage.  }\label{fig:theory:phase_current}
\end{figure}

The resultant current \textit{I} drawn from the voltage source can now be expressed as 
\begin{equation}
   \begin{aligned}\label{eq:theory:complex_current}
I = |I|e^{j(\omega t - \varphi)}
\end{aligned} 
\end{equation}
where $|I|$, $\omega$ and $\varphi$ respectively are  the magnitude angular frequency and phase shift phase shift of the current. It is common to also express the complex current as

\begin{equation}
   \begin{aligned}\label{eq:theory:complex_current_angular}
I = |I|\angle -\varphi
\end{aligned} 
\end{equation}

Inductive and capacitive reactances can also be expressed as complex numbers. A coil phase shifts the current 90 degrees behind the voltage. Considering the current $I_{q}$ as a complex number in Ohm's law gives that the inductive reactance can be expressed as $jX_{L}$, because multiplication by \textit{j} corresponds to a 90 degree rotation in the complex plane

\begin{equation}
   \begin{aligned}\label{eq:theory:complex_rectance_capacitive}
U = I\cdot jX_{L}
\end{aligned} 
\end{equation}

Similarly, a capacitive reactance $X_{c}$ phase shifts the current 90 degrees a head of the current. Therefore, it can be expressed as $-jX_{c}$

\begin{equation}
   \begin{aligned}\label{eq:theory:complex_rectance_inductance}
U = I\cdot (-jX_{c})
\end{aligned} 
\end{equation}

The complex notation also works for a circuit with resistive, inductive and capacitive components connected in series, as shown in figure \ref{fig:theory:complex_impedance_series}.

\begin{figure}[ht!]
    \center
    \subimport{../}{circuits/complex_impedance_series.tex}
    \caption[size = 9]
    {AC circuit with a resistor, capacitor and coil connected in series}\label{fig:theory:complex_impedance_series}
\end{figure}

The sum of the resistance and reactances is called the impedance $Z$ of the circuit and is given as

\begin{equation}
   \begin{aligned}\label{eq:theory:complex_impedance}
Z = R + jX_{L} - jX_{c}
\end{aligned} 
\end{equation}
Using the complex impedance \textit{Z} in Ohm's law describes both the resultant magnitude $|I|$and phase angle of the current $\varphi$ 



The observation making it easy to work with complex numbers in an AC power system is that the angular frequency $\omega$ of the voltages are the same. The frequency of a generator's voltage is directly proportional to the rotational speed of the turbine. If a generator in power system tries to gain speed, it will naturally be slowed down by the other generators in the system, bringing all of the different voltage to the same frequency. Adding two synchronous signals will give a new signal that inherits their frequency. There will be some phase shift to the new signal, but it does not depend on the frequency. As a results, it is possible to treat the sinusoidal signals (such as voltage and current) as a number in the complex plane.  

where \textit{U} is the voltage difference between the two buses which the line is connecting, \textit{Z} is the impedance of the line and \textit{I} is the current flowing in the line. The impedance of a power line is mainly inductive, which results in phase shift between the sinusoidal  current and the voltage difference. 


\section{Components in the power system}
An electrical power system consists of a set of nodes which can be thought of as the connection points in the system. These nodes are commonly referred to as buses. The connections or branches between the buses are power lines, cables, transformers or other power electronics equipment. The buses and branches defines the topology of the electrical power system. 

\subsection{Types of buses}
There are three types of buses in a power system\cite{opf_intro}.


\begin{itemize}
  \item Slack bus / reference bus
\end{itemize}
Its voltage angle is defined to be 0 and the angles at other buses are relative to the slack bus. There is only one slack bus in an electrical power system.

\begin{itemize}
  \item Load bus
\end{itemize}
A load bus is the most common bus in an electrical grid. The load buses can not control the flow of active and reactive power, as this is predetermined by the demand in the marked. As a result, it is also called PQ bus.

\begin{itemize}
  \item Voltage controlled bus
\end{itemize}
The voltage magnitude and active power are known for the voltage controlled bus, while the voltage angle and reactive power can vary.



\section{Two bus system}




\begin{figure}[ht!]
    \center
    \subimport{../}{circuits/two_bus.tex}
    \caption[size = 9]
    {Simple two bus system connected by a line. \textit{P} and \textit{Q} are the active and reactive power flow, \textit{R} and \textit{X} is the resistance and reactance of the line, \textit{U} is the voltage and \textit{I} is the current flowing}    \label{fig:theory:two_bus}
\end{figure}

Figure \ref{fig:theory:two_bus} displays power flow between to buses connected by a transmission line. From the left we have active power $P_{1}$ and reactive power $Q_{1}$ flowing. The power flows through the line and continues out from bus 2 as $P_{2}$ and $Q_{2}$. $U_{1}$ is a complex representation of the voltage at bus 1, $U_{1} = |U_{1}|e^{j\delta}$. Similarly, $U_{2} = |U_{2}|e^{j0} = |U_{2}|$. The relation between voltage and current for the system can be described by Ohm's law.

\begin{equation}\label{eq:chap1_ohm}
    U_{1} - U_{2} = (R + jX)I
\end{equation}
$R$ is the resistance and $X$ is the reactance of the line, $U_{1}$ and $U_{2}$ are the voltages at bus 1 and 2, and \textit{I} is the current flowing in the line. The impedance \textit{Z} of the line is also a complex number that can be expressed as $Z = |Z|e^{j\epsilon}$ where $tan (\epsilon) = X/R$. The current \textit{I} is commonly defined to be lagging, so that $I = |I|e^{-j\varphi}$. By this definition, $\varphi$ is a positive real number when the current is lagging the voltage. In figure \ref{fig:two_bus_phasor}, the current, voltages and impedance are drawn as phasors in the complex plane for a line with zero resistance (R=0).


\begin{figure}[ht!]
    \center
    \includegraphics[width=10cm]{figures/two_bus_phasor.PNG}
    \caption[size = 9]{Phasors of current and voltages in a two-port network connected by line with 0 resistance (R=0)}
	\label{fig:two_bus_phasor}
\end{figure}

The apparent power $S$ flowing in a line is defined to be
\begin{equation}\label{eq:apparent_power}
    S  = UI^{*} = P + jQ
\end{equation}

Where \textit{U} is the voltage, $I^{*}$ is the complex conjugate of the current, \textit{P} is the active power and \textit{Q} is the reactive power. The conjugation of the current is a convenience to make the reactive power \textit{Q} a positive number when the current is lagging the voltage. Using using \eqref{eq:chap1_ohm}, the current \textit{I} in the line can be expressed as

\begin{equation}\label{eq:two_port_current}
    I  = \frac{U_{1} - U_{2}}{Z}
    = \frac{|U_{1}|e^{j\delta} - |U_{2}|}{|Z|e^{j\epsilon}}
    = \frac{|U_{1}|e^{j(\delta- \epsilon)} - |U_{2}|e^{-j\epsilon}}{|Z|}
\end{equation}
where $|U_{1}|$ and $|U_{2}|$ are the voltage magnitudes at bus 1 and 2, \textit{I} is the current flowing in the line and \textit{|Z|} is the impedance magnitude for the line. Using the definition of power flow in \eqref{eq:apparent_power} on bus 1 gives that the active power $P_{1}$ is given by

\begin{equation}\label{eq:two_port_active_power}
P_{1} = \frac{|U_{1}|^2}{|Z|}cos(\epsilon) - \frac{|U_{1}||U_{2}|}{|Z|}cos(\epsilon - \delta)
\end{equation}
To make it easier to investigate a situation with active losses in the line (R = 0), it is convenient to introduce a loss angle $\alpha = tan(R/X)$. By the sum of angles in a triangle, we have the relation $\alpha + \epsilon = \pi/2$. By using the loss angle $\alpha$, the active power at bus 1 $P_{1}$ can be expressed as

\begin{equation}\label{eq:two_port_active_power_good}
P_{1} = \frac{|U_{1}|^2}{|Z|}sin(\alpha) + \frac{|U_{1}||U_{2}|}{|Z|}sin(\delta -\alpha)
\end{equation}
A line without resistive losses can now be examined simply by setting $\alpha = 0$. The resulting active power flow $P_{1}$ reduces to

\begin{equation}\label{eq:two_port_active_power_lossless}
P_{1} =  \frac{|U_{1}||U_{2}|}{|Z|}sin(\delta)
\end{equation}
where $|U_{1}|$ and $|U_{2}|$ are the voltage magnitudes at bus 1 and 2, $|Z|$ is the impedance magnitude of the line and $\delta$ is the phase angle at bus 1 with respect to bus 2. The takeaway from this analysis is that the direction of the active power flow is determined by the phase angle $\delta$ of the voltages between the buses. In a two bus system, the bus with the leading voltage is supplying active power, while the lagging voltage is receiving. Another thing to note is that for an AC power system the voltage magnitude between the buses may differ in size, and active power can still flow in both directions. This would not be possible in DC power system. Figure \ref{fig:theory:active_power_flow} shows the relation between a loss free power line and a sample line in pandapower. We see that the active power $P$ for a line with resistive losses is also mainly controlled by the voltage angle $\delta$. This is especially true around 0 voltage angle $\delta$, which is the region it generally is located. 


\begin{figure}[ht!]
    \center
    \subimport{../}{circuits/plot_active_power_flow.tex}
    \caption[size = 9]{Active power flow \textit{P} between two buses as a function of voltage angle with no losses ($\alpha=0$) a and typical loss angle $\alpha$ of 0.13} \label{fig:theory:active_power_flow}
\end{figure}

A similar argument will give that the reactive power flow $Q_{1}$ at bus 1 for a line without resistive losses is given by 

\begin{equation}\label{eq:two_port_reactive_power_lossless}
Q_{1} =  \frac{|U_{1}|}{|Z|}(|U_{1}| - |U_{2}|)
\end{equation}
where $|U_{1}|$ and $|U_{2}|$ are the voltage magnitudes at bus 1 and 2 respectively and $|Z|$ is the impedance magnitude of the line. The voltage angle $\delta$ is assumed to be small so that $cos(\delta) \approx 1$. Equation \eqref{eq:two_port_reactive_power_lossless} shows that the direction of the reactive power is determined by the difference of the voltage magnitudes. In other words, the bus with the highest voltage magnitude is supplying reactive power in a two bus system.

\section{Power flow in a grid}
Figure \ref{fig:power_flow_network} is an electrical model of bus \textit{k} in a power system that is connected to \textit{n} buses. $I_{k}$ is the net injected current to the grid from bus \textit{k}\cite{opf_intro}. It is common to work with the admittance $Y$ of a line, which is defined as the inverse of the impedance $Y = 1/Z$. By doing this, the current in a branch is easily found by multiplying the voltage difference and the admittance. The current flowing from bus \textit{k} to bus \textit{j} is $I_{j} = (U_{k} - U_{j})Y_{kj}$. The current flowing out of bus \textit{k} must equal the injected current $I_{k}$. This gives

\begin{equation}\label{eq:powerflow_currentsum}
I_{k} =  U_{k}Y_{k0}
+ \sum_{j=1,j\neq k}^{n}(U_{k} - U_{j})Y_{kj}
\end{equation}


\begin{figure}[ht]
    \center
    \includegraphics[width=8cm]{figures/power_flow_network.PNG}
    \caption[size = 9]{Electrical model of a bus connected to n buses}
	\label{fig:power_flow_network}
\end{figure}

This is a linear system of equation that can be expressed compactly as a matrix equation with the bus voltages ordered in a vector $U_{bus}$

\begin{equation}\label{eq:powerflow_busmatrix}
I_{bus} = Y_{bus}U_{bus}
\end{equation}
where $I_{bus}$ are n-dimensional vectors whose j-th component is the current flowing from \textit{k} to bus \textit{j}.


\section{Electric model of a power line}
A common way to model a transmission line is by using a $\pi$-equivalent circuit, as shown in figure \ref{fig:theory:PI_model}

\begin{figure}[ht!]
    \subimport{../}{circuits/pi_model.tex}
    \caption{$\pi$-equivalant model for a transmission line.}
    \label{fig:theory:PI_model}
\end{figure}



The transmission line is modelled by a resistance and inductance in series. In addition, there are two shunt capacitors connected in parallel at each end of the line. The capacitors are there because the flow of charges will give an associated capacitance that is proportional with the line length. The $\pi$-model splits the admittance $Y$ of the capacitor into two and puts one part on each end of the line. It is this model pandapower uses to solve a given transmission system. 



\end{document}
